Pour réaliser notre projet, nous pouvions utiliser tout un panel
de caractéristiques adaptées à la reconnaissance de formes. Ces 
caractéristiques permettent d'obtenir des informations sur chaque
échantillon et ainsi de créer une \og carte d'identité \fg pour
chaque chiffres. Il sera ainsi plus facile de comparer la base
d'apprentissage aux échantillons de la base de test. \\
Les caractéristiques proposées par M. Chatelain dans l'énoncé du
projet sont au nombre de deux. La première consiste en l'élaboration
de profils pour chaque imagette. C'est la première méthode que
nous avons décidé d'implémenter. Suite à une petite étude que nous
avons fait, nous avons constaté que la méthode était optimale pour un
nombre de profils de 10. \\
La seconde méthode consiste en l'étude de densité de pixels au sein de 
l'imagette. Encore une fois, nous avons déterminé que les meilleurs
paramètres pour l'utilisation de cette méthode consiste à diviser 
l'imagette en 10 rangées pour 5 colonnes. \\
Après extraction complète du modèle, celui-ci se trouve dans une 
matrice qui sauvegarde pour chaque imagette toute les caractéristiques.
Cette matrice est enregistrée dans le fichier \texttt{modeleRDF.mat}.